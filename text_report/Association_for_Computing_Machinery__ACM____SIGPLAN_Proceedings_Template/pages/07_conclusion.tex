\section{Conclusion}
\subsection{Summary of Findings}

\subsection{Future Work}
Future research should address several existing research gaps in the application of LLM-based pedagogical agents in education. While current studies demonstrate positive short-term effects on learning outcomes, there is still limited evidence regarding the long-term impact of these systems on knowledge retention, critical thinking development, and learner independence. 
\\
In addition, although social cues in affective pedagogical agents have been shown to improve engagement and motivation, there is a lack of research identifying the optimal design and level of human-likeness required for different learner groups and educational contexts. 
\\
Additionally, a notable gap in research is the scarcity of practical implementations in real classroom settings, as many experiments are still performed in controlled environments. Consequently, future research should prioritize studies conducted in actual educational contexts, create standardized assessment models for Human–AI educational technologies, and address ethical issues such as protecting data privacy, ensuring transparency, and promoting responsible AI usage to support a sustainable and reliable integration of technology in education.
