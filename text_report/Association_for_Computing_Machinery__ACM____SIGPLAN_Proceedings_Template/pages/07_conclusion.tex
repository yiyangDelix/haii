\section{Conclusion}

\subsection{Future Work}
Future research should address several existing research gaps in the application of LLM-based pedagogical agents in education. While current studies demonstrate positive short-term effects on learning outcomes, there is still limited evidence regarding the long-term impact of these systems on knowledge retention, critical thinking development, and learner independence. 
\\
In addition, although social cues in affective pedagogical agents have been shown to improve engagement and motivation, there is a lack of research identifying the optimal design and level of human-likeness required for different learner groups and educational contexts. 
\\
Additionally, a notable gap in research is the scarcity of practical implementations in real classroom settings, as many experiments are still performed in controlled environments. Consequently, future research should prioritize studies conducted in actual educational contexts, create standardized assessment models for Human–AI educational technologies, and address ethical issues such as protecting data privacy, ensuring transparency, and promoting responsible AI usage to support a sustainable and reliable integration of technology in education.


\subsection{Summary of Findings}
In conclusion, this study reveals that empathetic AI tutoring does not significantly impact learning outcomes, user sentiment, confusion levels, or response times when compared to neutral interaction modes.
\\
However, empathy plays a crucial role in shaping how students engage with learning systems. Students interacting with the Empathy Mode demonstrated markedly higher behavioral engagement—producing more words, initiating more conversational turns, and maintaining denser communication throughout their sessions.
\\
Conversely, the Neutral Mode encouraged greater diversity, though with reduced overall interaction volume. Temporal patterns further distinguished the two conditions: empathetic interactions sustained or deepened dialogue over time, while neutral interactions led to progressively shorter responses.
\\
Correlation analyses reinforced the interconnected nature of engagement metrics, yet revealed that time spent in the system bore little relationship to actual learning gains. While both groups showed positive trends in knowledge improvement from pre- to post-intervention, these changes did not reach statistical significance.
\\
Ultimately, these findings suggest that empathetic AI tutoring functions primarily as an engagement catalyst rather than a direct driver of measurable learning performance in short-term educational interventions.


