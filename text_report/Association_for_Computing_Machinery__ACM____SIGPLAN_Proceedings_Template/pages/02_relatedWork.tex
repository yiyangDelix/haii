\section{Related Work}
\subsection{Applications of LLMs in Education}
Education is essentially about knowledge transfer, instant feedback, and emotional interaction. LLMs mainly enhance the “immediate feedback" process in education. They have the potential to revolutionize the education industry by providing personalized, adaptive learning experiences for students.
\\
LLMs are shifting towards a more human-like approach, providing authentic conversational teaching experiences in various scenarios instead of simply giving answers. This is particularly noticeable when LLMs simulate a teacher’s role and ask questions to encourage critical thinking and independent exploration. By creating a self-learning environment, LLMs can help students develop their problem-solving skills and become more effective learners. Amongst others, a large meta-analysis in the {\itshape British Journal of Educational Technology} reports that AI chatbots have a positive effect on students’ learning outcomes.

\begin{figure}[h]
  \centering
  \includegraphics[width=\linewidth]{pic/eduPic.png}
  \caption{Characters of Education under LLMs}
  \label{fig:edu}
\end{figure}


\subsection{Pedagogical Virtual Characters and Human-AI Interaction}
The rationale for integrating social cues into AIPAs {\itshape (Affective Intelligent Pedagogical Agent)} is strongly supported by the principles of social agency theory (Moreno et al., 2001). Social agency theory emphasizes that when learners perceive instructional agents as social partners with human-like qualities, they are more likely to develop positive emotional experiences and learning motivation, thereby enhancing their learning outcomes. By incorporating social cues, AIPAs can make interactions with learners more natural and engage in deeper cognitive processing.
\\
Empirical research has reported that the social cues of AIPAs can detectably enhance student engagement (Schodde et al., 2019), motivation (Saerbeck et al., 2010), and academic performance (Zhang et al., 2024a). The findings showed that such affective pedagogical agents can promote learners' positive emotions, motivation, and overall academic outcomes.


\subsection{User Research and Learning Evaluation Methods}
Evaluation plays a central role in Human–AI Interaction (HAI) research, as it enables researchers to assess both system effectiveness and user experience. Building on established traditions in Human–Computer Interaction, HAI studies commonly employ quantitative, qualitative, and mixed-methods approaches to capture complementary aspects of human–AI systems.
\\
Quantitative evaluation methods are frequently used to measure learning outcomes and task performance. These include \textbf{pre- and post-tests}, controlled experiments, and standardized questionnaires, which allow for statistical comparison of user performance, perceived usability, and satisfaction. Foundational HCI literature emphasizes the reliability and scalability of such methods when evaluating interactive systems. In learning-oriented Human–AI systems, pre-/post-test designs are particularly effective for assessing knowledge acquisition and skill improvement.
\\
Qualitative methods are used to complement quantitative findings by providing deeper insight into user perceptions, reasoning processes, and contextual factors. Common techniques include semi-structured interviews, open-ended survey questions, and think-aloud protocols.
