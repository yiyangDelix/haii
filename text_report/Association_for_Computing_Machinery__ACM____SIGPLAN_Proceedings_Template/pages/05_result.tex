\section{Results}
\subsection{Overview of Data Analysis}
A total of 30 valid participant datasets were analyzed (Empathy Mode: n=15; Neutral Mode: n=15). \textbf{Table~\ref{tab:learning_outcomes}} presents the comprehensive descriptive statistics and independent samples t-test results for all measured variables, categorized by learning outcomes, interaction engagement, and cognitive metrics.
\\
In terms of learning effectiveness, \textbf{no significant difference was observed} between the Empathy Mode (M = 7.00, SD = 2.27) and the Neutral Mode (M = 6.93, SD = 2.94), t(28) = 0.07, p = .945. Similarly, users in both conditions reported comparable levels of sentiment (p = .682) and confusion rate (p = .610). The total interaction duration and average response time also showed no statistically significant variations between the two groups (see Table~\ref{tab:learning_outcomes}).
\\
However, analyses revealed \textbf{significant disparities} in behavioral engagement and interaction quality. As shown in Table 1, the Empathy Mode elicited a substantially higher volume of user output (p < .001) and communication density (p < .001). In contrast, the Neutral Mode demonstrated significantly higher Lexical Diversity (p < .001). These significant findings are visualized and detailed in the following subsections.

\begin{table*}[t]
\centering
\caption{Comparison of Learning, Interaction, and User Experience Metrics Between Empathy and Neutral Modes}
\label{tab:learning_outcomes}
\label{tab:empathy_vs_neutral}
\footnotesize
\setlength{\tabcolsep}{5pt}
\begin{tabular}{lccccc}
\toprule
\textbf{Measure} 
& \textbf{Empathy Mode} 
& \textbf{Neutral Mode} 
& \textbf{$t$} 
& \textbf{$p$} 
& \textbf{Cohen's $d$} \\
& \textbf{Mean (SD)} 
& \textbf{Mean (SD)} 
&  &  &  \\
\midrule
\multicolumn{6}{l}{\textit{Learning Outcomes}} \\
Learning Score 
& 7.00 (2.27) & 6.93 (2.94) & 0.07 & .945 & 0.03 \\
Confusion Rate 
& 0.02 (0.04) & 0.01 (0.02) & 0.52 & .610 & 0.19 \\

\midrule
\multicolumn{6}{l}{\textit{Interaction Quantity}} \\
Total Word Count 
& 181.13 (38.10) & 31.67 (11.13) & 14.58 & $<.001^{***}$ & 5.33 \\
Turn Count 
& 27.93 (4.15) & 23.40 (2.38) & 3.67 & .001$^{**}$ & 1.34 \\
Total Duration (s) 
& 701.27 (271.67) & 904.87 (347.25) & -1.79 & .085 & -0.65 \\

\midrule
\multicolumn{6}{l}{\textit{Interaction Quality \& Cognition}} \\
Communication Density (Words/Min) 
& 16.98 (5.54) & 2.46 (1.59) & 9.75 & $<.001^{***}$ & 3.56 \\
Lexical Diversity (TTR) 
& 0.19 (0.07) & 0.41 (0.15) & -5.24 & $<.001^{***}$ & -1.92 \\
Average Response Time (s) 
& 18.65 (6.81) & 24.70 (12.84) & -1.61 & .122 & -0.59 \\
Questions Asked 
& 1.00 (2.10) & 0.20 (0.41) & 1.44 & .169 & 0.53 \\

\midrule
\multicolumn{6}{l}{\textit{User Experience}} \\
Sentiment Score 
& 7.93 (2.34) & 8.27 (2.05) & -0.41 & .682 & -0.15 \\
\bottomrule
\end{tabular}
\end{table*}


\subsection{Interaction Quantity and Duration}
\begin{figure}[h]
  \centering
  \includegraphics[width=\linewidth]{pic/01WordCount.png}
  \caption{Word Count}
  \label{fig:wordcount}
\end{figure}

The analysis of interaction quantity revealed significant disparities between the two groups. As illustrated in Figure~\ref{fig:wordcount}, the \textbf{Total User Word Count} in the Empathy Mode (M = 181.13, SD = 38.10) was significantly higher than that in the Neutral Mode (M = 31.67, SD = 11.13), t(28) = 14.58, p < .001. This represents a substantial effect size (Cohen's d = 5.33), indicating that users in the Empathy condition generated nearly six times more text than those in the Neutral condition.
\\
Similarly, the \textbf{Turn Count (Figure~\ref{fig:turncount}} was significantly higher for the Empathy group (M = 27.93, SD = 4.15) compared to the Neutral group (M = 23.40, SD = 2.38), t(28) = 3.67, p = .001. Regarding time investment, \textbf{Total Duration (Figure~\ref{fig:duration})} showed a different trend. Although the Neutral group recorded a longer average duration (M = 904.87s) compared to the Empathy group (M = 701.27s), this difference was not statistically significant (t = -1.79, p = .085), and the Neutral group exhibited a notably larger standard deviation (SD = 347.25), suggesting high variability in user dwell time.

\begin{figure}[h]
  \centering
  \includegraphics[width=\linewidth]{pic/02TurnCount.png}
  \caption{Turn Count}
  \label{fig:turncount}
\end{figure}


\begin{figure}[h]
  \centering
  \includegraphics[width=\linewidth]{pic/03Duration.png}
  \caption{Duration}
  \label{fig:duration}
\end{figure}

\subsection{Interaction Quality and Linguistic Patterns}
To investigate the nature of user responses, we analyzed Communication Density and Lexical Diversity. \textbf{Figure~\ref{fig:communicationdensity}} demonstrates a significant difference in \textbf{Communication Density}, defined as user words per minute. The Empathy Mode elicited a much higher density (M = 16.98 WPM) compared to the Neutral Mode (M = 2.46 WPM), t(28) = 9.75, p < .001, indicating a more rapid and fluid exchange of information.
\\
In contrast, as shown in \textbf{Figure~\ref{fig:ttr}}, the \textbf{Lexical Diversity (TTR)} presented an inverse pattern. The Neutral Mode showed a significantly higher TTR (M = 0.41, SD = 0.15) than the Empathy Mode (M = 0.19, SD = 0.07), t(28) = -5.24, p < .001. This metric indicates that while the Neutral group produced less total text, the vocabulary used was proportionately more diverse, whereas the Empathy group's extensive output contained more repetitive linguistic structures.

\begin{figure}[h]
  \centering
  \includegraphics[width=\linewidth]{pic/04CommunicationDensity.png}
  \caption{Communication Density}
  \label{fig:communicationdensity}
\end{figure}

\begin{figure}[h]
  \centering
  \includegraphics[width=\linewidth]{pic/05LexicalDiversity_TTR.png}
  \caption{Lexical Diversity (TTR)}
  \label{fig:ttr}
\end{figure}


\subsection{Process Trajectory}
\begin{figure}[h]
  \centering
  \includegraphics[width=\linewidth]{pic/06SentimentTrajectoryperTurn.png}
  \caption{Sentiment Trajectory per Turn}
  \label{fig:sentimentTrajectoryperTurn}
\end{figure}

The temporal evolution of the interaction was analyzed to understand user engagement over the course of the session. \textbf{Figure~\ref{fig:sentimentTrajectoryperTurn} (Sentiment Trajectory)} displays the sentiment polarity for each turn. Both groups maintained positive sentiment throughout the session, with the Empathy group showing a slightly more stable positive trend, although the overall mean sentiment scores did not differ significantly (p = .682).

\begin{figure}[h]
  \centering
  \includegraphics[width=\linewidth]{pic/07DialogueDepthEvolution.png}
  \caption{Dialogue Depth Evolution}
  \label{fig:dialogueDepthEvolution}
\end{figure}
\textbf{Figure~\ref{fig:dialogueDepthEvolution} (Dialogue Depth Evolution)} illustrates the average word count per turn across the timeline. A distinct divergence is observable: the Empathy group (represented by the red line) maintained or increased their word count per turn as the dialogue progressed, signifying sustained engagement. Conversely, the Neutral group (blue line) exhibited a flat or declining trend, often reverting to brief responses after the initial turns.

\subsection{Correlation Analysis}
\begin{figure}[h]
  \centering
  \includegraphics[width=\linewidth]{pic/08CorrelationHeatmap.png}
  \caption{Correlation Heatmap}
  \label{fig:correlationHeatmap}
\end{figure}
A Pearson correlation analysis was conducted to examine the relationships between the measured variables, as visualized in the \textbf{Heatmap (Figure~\ref{fig:correlationHeatmap})}. A strong positive correlation was observed between \textbf{User Word Count} and \textbf{Turn Count} (r > .80), as well as between \textbf{User Word Count} and \textbf{Communication Density} (r > .90). Notably, Total Duration showed a weak to negligible correlation with \textbf{Learning Score} ($r \approx 0.30$), suggesting that merely spending more time in the system did not directly translate to higher test scores. Furthermore, \textbf{Lexical Diversity} was negatively correlated with \textbf{User Word Count} ($r  \approx -0.85$), confirming that longer interactions tended to result in lower type-token ratios due to the natural repetition of function words in conversational speech.

\subsection{Knowledge Acquisition}
To evaluate the overall effectiveness of the educational intervention, we compared the aggregated knowledge accuracy between the Pre-survey (N = 61) and Post-survey (N = 32) phases. Due to the anonymous nature of the survey collection which prevented paired data matching, a group-level analysis was conducted using an independent samples t-test.
\begin{table}[h]
\centering
\caption{Comparison of Knowledge Accuracy Between Pre-Survey and Post-Survey}
\label{tab:pre_post_accuracy}
\footnotesize
\setlength{\tabcolsep}{2pt}
\begin{tabular}{lccccccc}
\toprule
\textbf{Group} 
& \textbf{N} 
& \textbf{Mean Accuracy (\%)} 
& \textbf{SD (\%)} 
& \textbf{$t$} 
& \textbf{df} 
& \textbf{$p$} 
& \textbf{Cohen's $d$} \\
\midrule
Pre-Survey 
& 61 & 69.18 & 32.16 & -1.35 & 70.82 & .181 & 0.29 \\
Post-Survey 
& 32 & 78.13 & 29.34 &  &  &  &  \\
\bottomrule
\end{tabular}
\end{table}
\\
Table~\ref{tab:pre_post_accuracy} presents the descriptive statistics and t-test results for the knowledge assessment. The Post-survey group demonstrated a higher mean accuracy (M = 78.13\%, SD = 29.34\%) compared to the Pre-survey baseline (M = 69.18\%, SD = 32.16\%). Although this difference did not reach statistical significance (t(70.82) = -1.35, p = .181), the analysis revealed a small-to-medium effect size (Cohen's d = 0.29), suggesting a positive trend in knowledge retention following the avatar-based learning session.