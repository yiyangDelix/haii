\section{Introduction}
\subsection{Scope and Rationale}
Recent advances in large language models (LLMs) have significantly expanded the potential of AI-supported learning systems. Beyond providing factual answers, modern LLM-based systems can engage in interactive dialogue, adapt explanations to learners’ needs, and simulate human-like instructional behaviors. These capabilities have positioned AI tutors as promising tools for personalized and scalable education.
\\
However, effective learning is not solely a cognitive process; it is also influenced by social and emotional factors such as perceived empathy, encouragement, and motivation. Prior research in Human–AI Interaction and educational psychology suggests that learners’ perceptions of an instructor’s social presence and affective behavior can substantially shape their learning experience.
\\
As AI systems increasingly take on instructional roles, understanding how different AI teaching styles influence learners becomes a critical design question.
\\
Within this context, the present project focuses on the design and evaluation of an avatar-based learning system that integrates an LLM-driven conversational tutor with a visual avatar. The system is designed to present psychological learning content through interactive dialogue while varying the avatar’s teaching personality. Specifically, the study compares an empathic teaching mode with a neutral teaching mode to examine how differences in emotional tone and instructional style affect learners’ experiences.
\\
By combining an interactive learning application with a controlled user study, this project aims to contribute empirical insights into the design of human-centered AI tutors. In particular, it seeks to clarify whether and how emotionally expressive AI avatars can enhance learners’ perceived learning effectiveness and overall learning experience.


\subsection{Challenges and Research Motivation}
Despite the growing adoption of LLMs in educational contexts, several challenges remain. One major challenge is the limited understanding of how emotional and social cues embedded in AI tutors influence learners’ perceptions and learning outcomes. While LLMs can generate fluent and contextually relevant responses, their instructional effectiveness depends heavily on how learners interpret the system’s behavior, tone, and responsiveness.
\\
Another challenge lies in the design of AI tutors that balance informational clarity with emotional appropriateness. Overly neutral systems may appear cold or unengaging, whereas overly expressive systems risk being perceived as distracting or inauthentic. Designing AI teaching personalities that support learning without overwhelming the learner is therefore a non-trivial task.
\\
Motivated by these challenges, this study explores the role of AI avatar personality as a key design variable in AI-supported learning environments. Drawing on theories from social agency and affective computing, the project investigates whether empathic behaviors such as encouragement, supportive language, and acknowledgment of learner difficulties can positively influence learners’ perceived learning effectiveness compared to a neutral instructional style.
\\
By empirically examining these effects, the project aims to inform future design guidelines for educational AI systems, particularly those that incorporate embodied or avatar-based interfaces.


\subsection{Project Goals and Research Questions}
The primary goal of this project is to design and evaluate an LLM-based, avatar-supported learning system and to examine how different AI teaching personalities influence learners’ experiences. Specifically, the study investigates how empathic versus neutral AI teaching styles affect perceived empathy, trust, motivation, and engagement, as well as learning outcomes. By combining an interactive learning application with user studies, the project seeks to contribute empirical insights to the design of human-centered AI tutors.
\\
Overall, the study aims to answer the broader research question:
\\
How does avatar empathy influence learner engagement and interaction behavior in an AI-supported educational environment?

