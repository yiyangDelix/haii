\section{Methodology}
\subsection{Research Design}
This study employed a within-subjects, pretest–posttest experimental design to evaluate the effectiveness of an AI-based avatar learning environment and to assess users’ perceived empathy and support during the interaction. All participants went through three consecutive stages: completing a pre-survey, engaging with the avatar-based learning experiment, and subsequently completing a post-survey. With this approach, it was possible to compare the participants’ knowledge, perceptions, and experiences before and after going through the learning environment.
\\
The experiment focused on examining both \textbf{cognitive outcomes} (learning effectiveness) and \textbf{affective outcomes} (perceived empathy, emotional support, and engagement), which are central to evaluating intelligent tutoring systems and conversational agents in educational contexts.
\subsection{Participants}
A total of \textbf{30 participants} completed all stages of the experiment and were included in the analysis. Participants were recruited through convenience sampling among university peers. They came from a very diverse educational background ranging from business management and engineering to environmental sciences. The majority of participants were between \textbf{21 and 29 years old}, reflecting a typical student population. The sample shared male and female population quite equally, approximately \textbf{63\% identified as women}, with the remaining participants identifying as men. Almost all participants reported prior experience with artificial intelligence tools, such as ChatGPT, Gemini, Perplexity, etc.
\\
Participants were informed about the purpose of the study and voluntarily agreed to take part in the experiment. The data was handled anonymously using random user IDs. This way it was possible to adhere to ethical considerations.

\subsection{Experiment and Survey Design}
The whole experiment consisted of three main components: pre-survey, avatar learning environment and post–survey. Below is the breakdown of each section.

\subsubsection{Pre-Survey}
The pre-survey was conducted using \textbf{Google Forms} and consisted of four main components:
\begin{itemize}
\item[(1)] \textbf{Demographics}, including age, gender, educational background and prior experience with AI-based tools.
\item[(2)] \textbf{Baseline knowledge assessment} related to the psychology concepts addressed in the learning environment.
\item[(3)] \textbf{Learning preferences and expectations} toward AI-supported learning experience. in terms of the type of responses or feedback people want to get.
\end{itemize}
These measures were used to establish a baseline for both previous knowledge and user perceptions prior to interacting with the avatar. The exact questions can be found in the Appendix.

\subsubsection{Avatar Learning Environment}
After completing the pre-survey, participants were instructed to access the LLM-powered avatar-based learning environment through a Streamlit application. In prior, participants were not told if the avatar would be empathic or neutral to avoid creating an expectation of the interaction type or have them behave differently based on the condition. The avatar then led the user through the learning session by asking a set of teaching questions relating to introductory psychology concepts and providing explanations in response to the users’ inputs.
\\
The interaction with the avatar was a dialog-based conversational one. The avatar continued to provide explanation of ideas and concepts until the participant demonstrated knowledge of the explained concept in their responses. There was no specific time limit to complete the tasks in the experiment which allowed the participants to interact with the content at their own pace. When the whole learning content had been presented, participants were asked to complete another short quiz at the end of the interaction. The difference in the learning effectiveness and the user's perception was analyzed by comparing the pre-survey with the post-survey.


\subsubsection{Post-Survey}
Following the interaction, participants completed a post-survey assessing the avatar learning environment and their subjective experience on the following:

\begin{itemize}
\item[(1)] \textbf{Post-intervention knowledge}, using the same pre-survey knowledge questions for a proper and fair assessment of the knowledge gained after the experiment.

\item[(2)] \textbf{Perceived learning effectiveness}, including clarity, usefulness, and engagement of the avatar.

\item[(3)] \textbf{Perceived empathy and emotional support}, the degree to which the avatar was perceived as understanding and supportive.

\item[(4)] \textbf{Overall user experience}, including satisfaction, trust, and perceived value of the avatar as a learning tool.

\end{itemize}
Most questions were measured using a \textbf{5 point Likert-scale} system to allow for quantitative comparison across participants during the analysis stage. The exact questions can be found in the Appendix.

\subsubsection{Research Question and Hypotheses}
This research mainly focused on examining how the avatar empathy influences learner engagement and interaction behavior in an AI-supported educational environment. The following hypotheses will be addressed throughout this project work.
\\
\textbf{H1:} Learners interacting with an empathic avatar will demonstrate higher behavioral engagement than learners interacting with a neutral avatar.
\\
\textbf{H2:} Learners interacting with an empathic avatar will maintain deeper and more sustained interaction trajectories across the learning session compared to learners interacting with a neutral avatar.
