\documentclass[sigplan,screen]{acmart}

% ---- Disable ACM metadata ----
\settopmatter{printacmref=false}
\settopmatter{printfolios=false}

% ---------- FULL CLEAN ACM METADATA ----------
\settopmatter{
  printacmref=false,
  printccs=false,
  printfolios=false
}

\renewcommand\footnotetextcopyrightpermission[1]{}

\setcopyright{none}

\usepackage{fancyhdr}
\pagestyle{plain}
\fancyhf{}


\begin{document}

%%
%% The "title" command has an optional parameter,
%% allowing the author to define a "short title" to be used in page headers.
\title{Empathy in the Machine: How Avatar Personalities Shape Human Learning Experience}

%%
\author{Araks Karapetyan}
\affiliation{%
  \institution{Technical University Munich}
}
\email{akarapetyan533@gmail.com}

\author{Brishila Firza}
\affiliation{%
  \institution{Technical University Munich}}
\email{brishilafirza@gmail.com}

\author{Nergis Bilge}
\affiliation{%
  \institution{Technical University Munich}
}
\email{nergisbilge@gmail.com}

\author{Yiyang Xie}
\affiliation{%
 \institution{Technical University Munich}
}
\email{ge59jev@mytum.de}

\author{Yusong Yang}
\affiliation{%
  \institution{Technical University Munich}
}
\email{yusong.yang.tum@gmail.com}

%%
%% The abstract is a short summary of the work to be presented in the
%% article.
\begin{abstract}
  This project presents Psychology Learning Experiment, an AI-supported 
  educational system that combines large language models (LLMs) to 
  investigate human–AI interaction in psychology learning contexts. 
  The system is designed to demonstrates how prompt-based LLM control 
  and avatar-supported interfaces can be combined into a reproducible 
  experimental platform for studying learner–AI interactions, and it 
  provides design insights for future empathetic AI tutoring systems.
\end{abstract}

%%
%% Keywords. The author(s) should pick words that accurately describe
%% the work being presented. Separate the keywords with commas.
\keywords{AI Tutors; Large Language Models; Human–AI Interaction; 
Prompt Engineering; User Study; Educational Technology; Psychology Education}
%% A "teaser" image appears between the author and affiliation
%% information and the body of the document, and typically spans the
%% page.

%%
%% This command processes the author and affiliation and title
%% information and builds the first part of the formatted document.
\maketitle
% ---- Disable ACM metadata ----
\settopmatter{printacmref=false}
\settopmatter{printfolios=false}
% ---------- FULL CLEAN ACM METADATA ----------
\settopmatter{
  printacmref=false,
  printccs=false,
  printfolios=false
}
\renewcommand\footnotetextcopyrightpermission[1]{}
\setcopyright{none}
\pagestyle{plain}
\fancyhf{}


\section{Introduction}
\subsection{Scope and Rationale}
Recent advances in large language models (LLMs) have significantly expanded the potential of AI-supported learning systems. Beyond providing factual answers, modern LLM-based systems can engage in interactive dialogue, adapt explanations to learners’ needs, and simulate human-like instructional behaviors. These capabilities have positioned AI tutors as promising tools for personalized and scalable education.
\\
However, effective learning is not solely a cognitive process; it is also influenced by social and emotional factors such as perceived empathy, encouragement, and motivation. Prior research in Human–AI Interaction and educational psychology suggests that learners’ perceptions of an instructor’s social presence and affective behavior can substantially shape their learning experience.
\\
As AI systems increasingly take on instructional roles, understanding how different AI teaching styles influence learners becomes a critical design question.
\\
Within this context, the present project focuses on the design and evaluation of an avatar-based learning system that integrates an LLM-driven conversational tutor with a visual avatar. The system is designed to present psychological learning content through interactive dialogue while varying the avatar’s teaching personality. Specifically, the study compares an empathic teaching mode with a neutral teaching mode to examine how differences in emotional tone and instructional style affect learners’ experiences.
\\
By combining an interactive learning application with a controlled user study, this project aims to contribute empirical insights into the design of human-centered AI tutors. In particular, it seeks to clarify whether and how emotionally expressive AI avatars can enhance learners’ perceived learning effectiveness and overall learning experience.


\subsection{Challenges and Research Motivation}
Despite the growing adoption of LLMs in educational contexts, several challenges remain. One major challenge is the limited understanding of how emotional and social cues embedded in AI tutors influence learners’ perceptions and learning outcomes. While LLMs can generate fluent and contextually relevant responses, their instructional effectiveness depends heavily on how learners interpret the system’s behavior, tone, and responsiveness.
\\
Another challenge lies in the design of AI tutors that balance informational clarity with emotional appropriateness. Overly neutral systems may appear cold or unengaging, whereas overly expressive systems risk being perceived as distracting or inauthentic. Designing AI teaching personalities that support learning without overwhelming the learner is therefore a non-trivial task.
\\
Motivated by these challenges, this study explores the role of AI avatar personality as a key design variable in AI-supported learning environments. Drawing on theories from social agency and affective computing, the project investigates whether empathic behaviors such as encouragement, supportive language, and acknowledgment of learner difficulties can positively influence learners’ perceived learning effectiveness compared to a neutral instructional style.
\\
By empirically examining these effects, the project aims to inform future design guidelines for educational AI systems, particularly those that incorporate embodied or avatar-based interfaces.


\subsection{Project Goals and Research Questions}
The primary goal of this project is to design and evaluate an LLM-based, avatar-supported learning system and to examine how different AI teaching personalities influence learners’ experiences. Specifically, the study investigates how empathic versus neutral AI teaching styles affect perceived empathy, trust, motivation, and engagement, as well as learning outcomes. By combining an interactive learning application with user studies, the project seeks to contribute empirical insights to the design of human-centered AI tutors.
\\
Overall, the study aims to answer the broader research question:
\\
How does avatar empathy influence learner engagement and interaction behavior in an AI-supported educational environment?



\section{Related Work}
\subsection{Applications of LLMs in Education}
Education is essentially about knowledge transfer, instant feedback, and emotional interaction. LLMs mainly enhance the “immediate feedback" process in education. They have the potential to revolutionize the education industry by providing personalized, adaptive learning experiences for students.
\\
LLMs are shifting towards a more human-like approach, providing authentic conversational teaching experiences in various scenarios instead of simply giving answers. This is particularly noticeable when LLMs simulate a teacher’s role and ask questions to encourage critical thinking and independent exploration. By creating a self-learning environment, LLMs can help students develop their problem-solving skills and become more effective learners. Amongst others, a large meta-analysis in the {\itshape British Journal of Educational Technology} reports that AI chatbots have a positive effect on students’ learning outcomes.

\begin{figure}[h]
  \centering
  \includegraphics[width=\linewidth]{pic/eduPic.png}
  \caption{Characters of Education under LLMs}
  \label{fig:edu}
\end{figure}


\subsection{Pedagogical Virtual Characters and Human-AI Interaction}
The rationale for integrating social cues into AIPAs {\itshape (Affective Intelligent Pedagogical Agent)} is strongly supported by the principles of social agency theory (Moreno et al., 2001). Social agency theory emphasizes that when learners perceive instructional agents as social partners with human-like qualities, they are more likely to develop positive emotional experiences and learning motivation, thereby enhancing their learning outcomes. By incorporating social cues, AIPAs can make interactions with learners more natural and engage in deeper cognitive processing.
\\
Empirical research has reported that the social cues of AIPAs can detectably enhance student engagement (Schodde et al., 2019), motivation (Saerbeck et al., 2010), and academic performance (Zhang et al., 2024a). The findings showed that such affective pedagogical agents can promote learners' positive emotions, motivation, and overall academic outcomes.


\subsection{User Research and Learning Evaluation Methods}
Evaluation plays a central role in Human–AI Interaction (HAI) research, as it enables researchers to assess both system effectiveness and user experience. Building on established traditions in Human–Computer Interaction, HAI studies commonly employ quantitative, qualitative, and mixed-methods approaches to capture complementary aspects of human–AI systems.
\\
Quantitative evaluation methods are frequently used to measure learning outcomes and task performance. These include \textbf{pre- and post-tests}, controlled experiments, and standardized questionnaires, which allow for statistical comparison of user performance, perceived usability, and satisfaction. Foundational HCI literature emphasizes the reliability and scalability of such methods when evaluating interactive systems. In learning-oriented Human–AI systems, pre-/post-test designs are particularly effective for assessing knowledge acquisition and skill improvement.
\\
Qualitative methods are used to complement quantitative findings by providing deeper insight into user perceptions, reasoning processes, and contextual factors. Common techniques include semi-structured interviews, open-ended survey questions, and think-aloud protocols.



\section{System Design}
The system supports text-based real-time interaction and reserves 
interfaces for future extensions such as voice input/output and 
more complex Avatar behaviors.

\subsection{Overall System Architecture}
The system adopts a modular, loosely coupled architectural design 
to support rapid prototyping, functional expansion, and subsequent 
user studies.

\subsubsection{Core technology components}

The main system consists of the following core components:

\begin{itemize}
\item {\textbf{Streamlit}}
\\
Used to build an interactive web interface, 
supporting rapid prototyping and user testing. Streamlit serves as the 
primary front-end entry point. It is responsible for receiving user 
input, displaying LLM responses, recording user IDs, redirecting to 
pre/post-test links, and interacting with other backend calling logic.

\item {\textbf{OpenAI API}}
\\
This project uses the gpt-4o-mini model 
as the core LLM for dialogue. The choice of gpt-4o-mini over models 
like GPT-3.5 or more advanced versions is primarily based on its 
balance of response speed, stability, and cost. This makes it suitable 
for the development and experimentation of an interactive teaching system.


\item {\textbf{Tiktoken}}
\\
Used for prompt management and token 
counting. It helps avoid exceeding model context limits by 
controlling context length, which also contributes to optimizing 
response speed and ensuring the stability of the teaching system.


\item {\textbf{Ready Player Me}}
\\
Used to generate a 3D visual 
Avatar prototype, providing a basic character model with facial 
expressions and body behaviors for the teaching system. This 
Avatar primarily aims to enhance the presence and interactive 
feel of the AI teacher within the system.

\item {\textbf{Pandas}}
\\
Used for structured processing of experimental data locally, 
including the organization of conversation logs, learning 
metrics, and statistical features.


\item {\textbf{Requests}}
\\
Used for communication with external services, e.g., 
redirection via Google Forms links and API requests.


\item {\textbf{Google Sheets API}}
\\
(gspread + google-auth) Used for automatically synchronizing 
experimental data to 
online spreadsheets in Google Drive. This method supports 
multi-person collaboration, real-time updates, and 
subsequent data analysis.

\item {\textbf{Edge-TTS and Mutagen (Early Version)}}
\\
Used in early versions of the system for experimental 
speech synthesis and audio processing, but later removed 
due to experimental control and stability issues.
\end{itemize}

\subsubsection{System Workflow Overview}
At the system level, the learning system supports a 
complete learning and data collection loop, including 
key stages such as pre-experiment questionnaire, interactive 
learning, and post-experiment data recording. The system 
connects to external questionnaire platforms via a web interface 
and continuously logs multi-dimensional data related to learning 
behaviors throughout the process.
\\
It should be noted that this section only provides an 
overview of the overall workflow from the perspectives 
of system functionality and technical support. Details 
regarding the specific experimental design, participant 
grouping, experimental steps, and questionnaire content 
will be explained in detail in 
the \textbf{Avatar Learning Environment} chapter.
\\
In the current system implementation, the system 
supports the following workflow capabilities:
\\
\begin{itemize}
\item {}
It can guide users to complete external questionnaires 
(e.g., Pre-Survey) before learning begins and generate 
a unique participant identifier (UUID) for data linkage 
without collecting personally identifiable information.

\item {}
It enables multi-turn text-based interaction powered 
by the LLM during the learning process and records 
learning behaviors and dialogue context in real-time.

\item {}
It supports exporting complete interaction data after 
the learning and automatically synchronizing one 
structured document (CSV and Google Sheets) for 
subsequent analysis.
\end{itemize}
By clearly distinguishing the system workflow, this 
project achieves a decoupling between system implementation 
and experimental methodology in its architectural design. 
This provides a foundation for the reproducibility and 
extensibility of future research.


\subsection{LLM Architecture and Prompt Design}
In an LLM based teaching system, maintaining role consistency, 
stability of teaching strategies, and controllability of dialogue 
is a key challenge. Unlike systems that rely solely on user 
input (User Prompt) to drive a model, this project employs the 
System Prompt as a core to implement teaching objectives, role 
definitions, and interaction rules.
\\
This section systematically introduces the design rationale for 
the System Prompt in this project and its application in 
teaching scenarios.

\subsubsection{Role Division of System/User/Assistant}
This project builds the LLM interaction logic based on the 
tripartite dialogue structure (System Prompt, User Prompt, 
Assistant Response) provided by OpenAI. Their responsibilities 
are divided as follows:

\begin{itemize}
\item {\textbf{System Prompt}}
\\
The System Prompt is an instruction provided by the system 
at the beginning of a conversation, used to define the LLM's 
overall behavior, tone, style, and interaction rules. In this 
project, the System Prompt primarily fulfills the following 
functions:

\begin{itemize}
\item {\textbf{Role Definition}}
\\
Clearly defines the LLM's 
identity in the teaching scenario, e.g., a psychology teacher.

\item {\textbf{Behavioral Constraints}}
\\
Specifies the scope 
of the model's responses, e.g., avoiding answers unrelated to 
psychology learning.


\item {\textbf{Context Provision}}
\\
Provides the model with 
background information about the teaching scenario and 
knowledge domain, making its responses better align with 
the expected learning objectives.

\item {\textbf{Output Specification}}
\\
Constrains the 
structure and style of the model's answers, e.g., 
emphasizing clarity of explanation or supportive feedback.
\end{itemize}

Unlike the User Prompt, the System Prompt is implicit within 
the LLM's internal processing and is not directly presented 
to the user. It serves as a stable, controllable technical 
foundation for the teaching system.

\item {\textbf{User Prompt}}
\\
The User Prompt represents the learner's input, which may 
include questions, answers, or reflective statements. Within 
the learning flow, the User Prompt is used to trigger 
different teaching phases, such as introducing new concepts, 
quizzes, or summaries.

\item {\textbf{Assistant Response}}
\\
Generated by the LLM based on both the System Prompt and 
the User Prompt, it is used to provide explanations, 
guidance, or feedback.
\end{itemize}

This structure allows teaching design to be directly embedded 
into the teaching system logic through Prompt Engineering.

\subsubsection{System Prompt Examples for Teaching Roles}
In this project, different teaching Avatars are differentiated 
through distinct System Prompts. For example:

\begin{itemize}
\item {\textbf{Supportive Avatar}}
\\
\begin{figure}[h]
  \centering
  \includegraphics[width=\linewidth]{pic/empathyProm.png}
  \caption{System Prompt for the Supportive Avatar}
\end{figure}
Its System Prompt emphasizes empathy, encouraging language, 
and guided questioning, aiming to create a relaxed and supportive 
learning atmosphere. {\itshape (Full examples are not shown in the 
main text due to space constraints.)}


\item {\textbf{Neutral Avatar}}
\\
Its System Prompt focuses on scientific explanation, conceptual 
accuracy, and objective feedback, minimizing emotional expression 
and highlighting information delivery. {\itshape (Full examples 
are not shown in the main text due to space constraints.)}

\begin{figure}[h]
  \centering
  \includegraphics[width=\linewidth]{pic/neutralProm.png}
  \caption{System Prompt for the Neutral Avatar}
\end{figure}

\end{itemize}


This method of role modeling based on System Prompts enables the 
construction of different experimental conditions by merely modifying 
the prompt conditions, while keeping the learning content consistent. 
This design provides a clear and controllable technical foundation 
for the comparative analysis of different modes in subsequent user studies.

\subsection{User Interface Design of the Learning System}
The UI of the system is designed to provide learners with an intuitive, 
relaxed learning experience that incorporates a sense of social presence, 
while maintaining experimental control. The overall interface follows 
the "Principle of Minimal Interference" \cite{software}, i.e., minimizing unrelated 
functions to avoid introducing variables that could affect learning 
and experimental results.

\subsubsection{Experiment Entry Interface}
Before entering the formal learning session, the system first presents 
an entry page designed to guide learners through completing a 
pre-experiment questionnaire (Pre-survey). The page header displays
a uniform greeting:

\begin{itemize}
\item {} {\itshape Welcome! Before we begin the session with the AI teacher, 
please complete a short survey.}
\end{itemize}

This prompt informs users of the experimental procedure, emphasizing 
to complete the questionnaire before interacting with the AI teacher. 
Furthermore, the interface provides clear operational instructions:
\begin{itemize}
\item {} {\itshape Please keep this tab open. After submitting the Google 
Form, return here and click the button below. And please keep all 
the pages open during the whole experiment.
}
\end{itemize}
These instructions help avoid data loss caused by users closing, 
refreshing, or navigating away from the page, thereby ensuring 
workflow continuity and data integrity.
\\
To enable data linkage between the Pre-test and Post-test, the 
system automatically generates and displays a unique Participant 
Identifier on this page, for example:
\begin{itemize}
\item {} {\itshape Your Participant ID: SUB-157e9d77 (Auto-filled) }
\end{itemize}
This auto-generated ID, shown to the user, allows for matching 
multi-stage data from the same participant without collecting 
personally sensitive information, thus preserving anonymity and 
traceability.
\begin{figure}[h]
  \centering
  \includegraphics[width=\linewidth]{pic/beginUI.png}
  \caption{Experiment Entry Interface}
\end{figure}
\\
After the Pre-survey, users can either click 
\begin{itemize}
\item {} {\itshape Click Here to Start Pre-Survey.}
\end{itemize}
to proceed, or, if they have already completed 
it, select
\begin{itemize}
\item {} {\itshape I have submitted the Pre-Survey.}
\end{itemize}
to enter the learning 
session directly. This design reduces user operational effort and 
minimizes experimental interruptions caused by unclear procedures.

\subsubsection{Learning Interface Layout}
Upon entering the learning system, the interface adopts a split-pane layout:
\begin{itemize}
\item {} Left Pane: 3D Avatar Display
\item {} Right Pane: Text-based AI Dialogue Window
\end{itemize}

The left pane features a 3D female Avatar. Based on prior 
research and surveys\cite{genderDiff} indicating that, compared to male or 
neutral figures, female teacher images can more easily help 
learners feel relaxed, establish trust, and maintain higher 
levels of concentration during learning. The primary purpose 
of this Avatar is to enhance the sense of social presence 
during learning, rather than to simulate complex emotions or behaviors.
\\
Users can interact with the Avatar via basic mouse controls—dragging 
to rotate the view and using the scroll wheel to zoom—adding a 
degree of explorability and immersion to the interface.
\\
The right pane contains a text dialogue window similar to those 
found in ChatGPT or Gemini. Learners can freely type questions, 
answer system-generated quiz items, or request further explanations 
within this window.
\\
\begin{figure}[h]
  \centering
  \includegraphics[width=\linewidth]{pic/InterUI.png}
  \caption{Learning Interface Layout}
\end{figure}
This design leverages a highly familiar interaction paradigm for 
large language models, lowering the learning curve and allowing 
users to focus their attention on the learning content itself 
rather than on interface mechanics.

\subsubsection{Functional Iteration and Design Trade-offs}
In earlier versions of the system, we experimented with features such as:
\begin{itemize}
\item {} Avatar speech output (TTS) and speech input (STT)
\item {} Teacher style selection (e.g., Supportive Avatar, Neutral Avatar)
\item {} A manual Participant ID entry window
\end{itemize}
However, these features were progressively removed during later 
development and testing for the following primary reasons:
\begin{figure}[h]
  \centering
  \includegraphics[width=30mm]{pic/manualID.png}
  \caption{Manual Participant ID Entry, Speech In- and Output}
\end{figure}
\begin{itemize}
\item {\textbf{Voice Feature Limitations}}
\\
The voice functionality proved highly sensitive to network 
conditions. Users experienced significant variations in latency. 
This unpredictability not only impacted user experience but also 
introduced uncontrolled variables for experimental results.

\item {\textbf{Decreasing Expectancy Effects}}
\\
Allowing users to actively choose a teacher style would make 
them explicitly aware of their experimental condition. 
This could influence subjective behavior. To better control 
variables and ensure experimental validity, this feature was 
deprecated.
\begin{figure}[h]
  \centering
  \includegraphics[width=30mm]{pic/selectModel.png}
  \caption{Teacher Style Selection}
\end{figure}

\item {\textbf{Streamlining Data Integrity}}
\\
The initial design requiring manual Participant ID entry. 
However, during testing, some users ignored the step 
entirely, leading to difficulties in data matching and 
increasing the complexity of subsequent data cleaning and 
analysis. This functionality was ultimately replaced by the 
auto-generated ID method to improve data consistency and 
experimental control.
\end{itemize}

The final interface design of this system strikes a balance 
between user experience, learning support, and experimental 
control, providing a stable and reliable foundation for data 
collection and analysis in the subsequent user study.


\section{Methodology}
\subsection{Research Design}
This study employed a within-subjects, pretest–posttest experimental design to evaluate the effectiveness of an AI-based avatar learning environment and to assess users’ perceived empathy and support during the interaction. All participants went through three consecutive stages: completing a pre-survey, engaging with the avatar-based learning experiment, and subsequently completing a post-survey. With this approach, it was possible to compare the participants’ knowledge, perceptions, and experiences before and after going through the learning environment.
\\
The experiment focused on examining both \textbf{cognitive outcomes} (learning effectiveness) and \textbf{affective outcomes} (perceived empathy, emotional support, and engagement), which are central to evaluating intelligent tutoring systems and conversational agents in educational contexts.
\subsection{Participants}
A total of \textbf{30 participants} completed all stages of the experiment and were included in the analysis. Participants were recruited through convenience sampling among university peers. They came from a very diverse educational background ranging from business management and engineering to environmental sciences. The majority of participants were between \textbf{21 and 29 years old}, reflecting a typical student population. The sample shared male and female population quite equally, approximately \textbf{63\% identified as women}, with the remaining participants identifying as men. Almost all participants reported prior experience with artificial intelligence tools, such as ChatGPT, Gemini, Perplexity, etc.
\\
Participants were informed about the purpose of the study and voluntarily agreed to take part in the experiment. The data was handled anonymously using random user IDs. This way it was possible to adhere to ethical considerations.

\subsection{Experiment and Survey Design}
The whole experiment consisted of three main components: pre-survey, avatar learning environment and post–survey. Below is the breakdown of each section.

\subsubsection{Pre-Survey}
The pre-survey was conducted using \textbf{Google Forms} and consisted of four main components:
\begin{itemize}
\item[(1)] \textbf{Demographics}, including age, gender, educational background and prior experience with AI-based tools.
\item[(2)] \textbf{Baseline knowledge assessment} related to the psychology concepts addressed in the learning environment.
\item[(3)] \textbf{Learning preferences and expectations} toward AI-supported learning experience. in terms of the type of responses or feedback people want to get.
\end{itemize}
These measures were used to establish a baseline for both previous knowledge and user perceptions prior to interacting with the avatar. The exact questions can be found in the Appendix.

\subsubsection{Avatar Learning Environment}
After completing the pre-survey, participants were instructed to access the LLM-powered avatar-based learning environment through a Streamlit application. In prior, participants were not told if the avatar would be empathic or neutral to avoid creating an expectation of the interaction type or have them behave differently based on the condition. The avatar then led the user through the learning session by asking a set of teaching questions relating to introductory psychology concepts and providing explanations in response to the users’ inputs.
\\
The interaction with the avatar was a dialog-based conversational one. The avatar continued to provide explanation of ideas and concepts until the participant demonstrated knowledge of the explained concept in their responses. There was no specific time limit to complete the tasks in the experiment which allowed the participants to interact with the content at their own pace. When the whole learning content had been presented, participants were asked to complete another short quiz at the end of the interaction. The difference in the learning effectiveness and the user's perception was analyzed by comparing the pre-survey with the post-survey.


\subsubsection{Post-Survey}
Following the interaction, participants completed a post-survey assessing the avatar learning environment and their subjective experience on the following:

\begin{itemize}
\item[(1)] \textbf{Post-intervention knowledge}, using the same pre-survey knowledge questions for a proper and fair assessment of the knowledge gained after the experiment.

\item[(2)] \textbf{Perceived learning effectiveness}, including clarity, usefulness, and engagement of the avatar.

\item[(3)] \textbf{Perceived empathy and emotional support}, the degree to which the avatar was perceived as understanding and supportive.

\item[(4)] \textbf{Overall user experience}, including satisfaction, trust, and perceived value of the avatar as a learning tool.

\end{itemize}
Most questions were measured using a \textbf{5 point Likert-scale} system to allow for quantitative comparison across participants during the analysis stage. The exact questions can be found in the Appendix.

\subsubsection{Research Question and Hypotheses}
This research mainly focused on examining how the avatar empathy influences learner engagement and interaction behavior in an AI-supported educational environment. The following hypotheses will be addressed throughout this project work.
\\
\textbf{H1:} Learners interacting with an empathic avatar will demonstrate higher behavioral engagement than learners interacting with a neutral avatar.
\\
\textbf{H2:} Learners interacting with an empathic avatar will maintain deeper and more sustained interaction trajectories across the learning session compared to learners interacting with a neutral avatar.


\section{Results}
\subsection{Overview of Data Analysis}
A total of 30 valid participant datasets were analyzed (Empathy Mode: n=15; Neutral Mode: n=15). \textbf{Table~\ref{tab:learning_outcomes}} presents the comprehensive descriptive statistics and independent samples t-test results for all measured variables, categorized by learning outcomes, interaction engagement, and cognitive metrics.
\\
In terms of learning effectiveness, \textbf{no significant difference was observed} between the Empathy Mode (M = 7.00, SD = 2.27) and the Neutral Mode (M = 6.93, SD = 2.94), t(28) = 0.07, p = .945. Similarly, users in both conditions reported comparable levels of sentiment (p = .682) and confusion rate (p = .610). The total interaction duration and average response time also showed no statistically significant variations between the two groups (see Table~\ref{tab:learning_outcomes}).
\\
However, analyses revealed \textbf{significant disparities} in behavioral engagement and interaction quality. As shown in Table 1, the Empathy Mode elicited a substantially higher volume of user output (p < .001) and communication density (p < .001). In contrast, the Neutral Mode demonstrated significantly higher Lexical Diversity (p < .001). These significant findings are visualized and detailed in the following subsections.

\begin{table*}[t]
\centering
\caption{Comparison of Learning, Interaction, and User Experience Metrics Between Empathy and Neutral Modes}
\label{tab:learning_outcomes}
\label{tab:empathy_vs_neutral}
\footnotesize
\setlength{\tabcolsep}{5pt}
\begin{tabular}{lccccc}
\toprule
\textbf{Measure} 
& \textbf{Empathy Mode} 
& \textbf{Neutral Mode} 
& \textbf{$t$} 
& \textbf{$p$} 
& \textbf{Cohen's $d$} \\
& \textbf{Mean (SD)} 
& \textbf{Mean (SD)} 
&  &  &  \\
\midrule
\multicolumn{6}{l}{\textit{Learning Outcomes}} \\
Learning Score 
& 7.00 (2.27) & 6.93 (2.94) & 0.07 & .945 & 0.03 \\
Confusion Rate 
& 0.02 (0.04) & 0.01 (0.02) & 0.52 & .610 & 0.19 \\

\midrule
\multicolumn{6}{l}{\textit{Interaction Quantity}} \\
Total Word Count 
& 181.13 (38.10) & 31.67 (11.13) & 14.58 & $<.001^{***}$ & 5.33 \\
Turn Count 
& 27.93 (4.15) & 23.40 (2.38) & 3.67 & .001$^{**}$ & 1.34 \\
Total Duration (s) 
& 701.27 (271.67) & 904.87 (347.25) & -1.79 & .085 & -0.65 \\

\midrule
\multicolumn{6}{l}{\textit{Interaction Quality \& Cognition}} \\
Communication Density (Words/Min) 
& 16.98 (5.54) & 2.46 (1.59) & 9.75 & $<.001^{***}$ & 3.56 \\
Lexical Diversity (TTR) 
& 0.19 (0.07) & 0.41 (0.15) & -5.24 & $<.001^{***}$ & -1.92 \\
Average Response Time (s) 
& 18.65 (6.81) & 24.70 (12.84) & -1.61 & .122 & -0.59 \\
Questions Asked 
& 1.00 (2.10) & 0.20 (0.41) & 1.44 & .169 & 0.53 \\

\midrule
\multicolumn{6}{l}{\textit{User Experience}} \\
Sentiment Score 
& 7.93 (2.34) & 8.27 (2.05) & -0.41 & .682 & -0.15 \\
\bottomrule
\end{tabular}
\end{table*}


\subsection{Interaction Quantity and Duration}
\begin{figure}[h]
  \centering
  \includegraphics[width=\linewidth]{pic/01WordCount.png}
  \caption{Word Count}
  \label{fig:wordcount}
\end{figure}

The analysis of interaction quantity revealed significant disparities between the two groups. As illustrated in Figure~\ref{fig:wordcount}, the \textbf{Total User Word Count} in the Empathy Mode (M = 181.13, SD = 38.10) was significantly higher than that in the Neutral Mode (M = 31.67, SD = 11.13), t(28) = 14.58, p < .001. This represents a substantial effect size (Cohen's d = 5.33), indicating that users in the Empathy condition generated nearly six times more text than those in the Neutral condition.
\\
Similarly, the \textbf{Turn Count (Figure~\ref{fig:turncount}} was significantly higher for the Empathy group (M = 27.93, SD = 4.15) compared to the Neutral group (M = 23.40, SD = 2.38), t(28) = 3.67, p = .001. Regarding time investment, \textbf{Total Duration (Figure~\ref{fig:duration})} showed a different trend. Although the Neutral group recorded a longer average duration (M = 904.87s) compared to the Empathy group (M = 701.27s), this difference was not statistically significant (t = -1.79, p = .085), and the Neutral group exhibited a notably larger standard deviation (SD = 347.25), suggesting high variability in user dwell time.

\begin{figure}[h]
  \centering
  \includegraphics[width=\linewidth]{pic/02TurnCount.png}
  \caption{Turn Count}
  \label{fig:turncount}
\end{figure}


\begin{figure}[h]
  \centering
  \includegraphics[width=\linewidth]{pic/03Duration.png}
  \caption{Duration}
  \label{fig:duration}
\end{figure}

\subsection{Interaction Quality and Linguistic Patterns}
To investigate the nature of user responses, we analyzed Communication Density and Lexical Diversity. \textbf{Figure~\ref{fig:communicationdensity}} demonstrates a significant difference in \textbf{Communication Density}, defined as user words per minute. The Empathy Mode elicited a much higher density (M = 16.98 WPM) compared to the Neutral Mode (M = 2.46 WPM), t(28) = 9.75, p < .001, indicating a more rapid and fluid exchange of information.
\\
In contrast, as shown in \textbf{Figure~\ref{fig:ttr}}, the \textbf{Lexical Diversity (TTR)} presented an inverse pattern. The Neutral Mode showed a significantly higher TTR (M = 0.41, SD = 0.15) than the Empathy Mode (M = 0.19, SD = 0.07), t(28) = -5.24, p < .001. This metric indicates that while the Neutral group produced less total text, the vocabulary used was proportionately more diverse, whereas the Empathy group's extensive output contained more repetitive linguistic structures.

\begin{figure}[h]
  \centering
  \includegraphics[width=\linewidth]{pic/04CommunicationDensity.png}
  \caption{Communication Density}
  \label{fig:communicationdensity}
\end{figure}

\begin{figure}[h]
  \centering
  \includegraphics[width=\linewidth]{pic/05LexicalDiversity_TTR.png}
  \caption{Lexical Diversity (TTR)}
  \label{fig:ttr}
\end{figure}


\subsection{Process Trajectory}
\begin{figure}[h]
  \centering
  \includegraphics[width=\linewidth]{pic/06SentimentTrajectoryperTurn.png}
  \caption{Sentiment Trajectory per Turn}
  \label{fig:sentimentTrajectoryperTurn}
\end{figure}

The temporal evolution of the interaction was analyzed to understand user engagement over the course of the session. \textbf{Figure~\ref{fig:sentimentTrajectoryperTurn} (Sentiment Trajectory)} displays the sentiment polarity for each turn. Both groups maintained positive sentiment throughout the session, with the Empathy group showing a slightly more stable positive trend, although the overall mean sentiment scores did not differ significantly (p = .682).

\begin{figure}[h]
  \centering
  \includegraphics[width=\linewidth]{pic/07DialogueDepthEvolution.png}
  \caption{Dialogue Depth Evolution}
  \label{fig:dialogueDepthEvolution}
\end{figure}
\textbf{Figure~\ref{fig:dialogueDepthEvolution} (Dialogue Depth Evolution)} illustrates the average word count per turn across the timeline. A distinct divergence is observable: the Empathy group (represented by the red line) maintained or increased their word count per turn as the dialogue progressed, signifying sustained engagement. Conversely, the Neutral group (blue line) exhibited a flat or declining trend, often reverting to brief responses after the initial turns.

\subsection{Correlation Analysis}
\begin{figure}[h]
  \centering
  \includegraphics[width=\linewidth]{pic/08CorrelationHeatmap.png}
  \caption{Correlation Heatmap}
  \label{fig:correlationHeatmap}
\end{figure}
A Pearson correlation analysis was conducted to examine the relationships between the measured variables, as visualized in the \textbf{Heatmap (Figure~\ref{fig:correlationHeatmap})}. A strong positive correlation was observed between \textbf{User Word Count} and \textbf{Turn Count} (r > .80), as well as between \textbf{User Word Count} and \textbf{Communication Density} (r > .90). Notably, Total Duration showed a weak to negligible correlation with \textbf{Learning Score} ($r \approx 0.30$), suggesting that merely spending more time in the system did not directly translate to higher test scores. Furthermore, \textbf{Lexical Diversity} was negatively correlated with \textbf{User Word Count} ($r  \approx -0.85$), confirming that longer interactions tended to result in lower type-token ratios due to the natural repetition of function words in conversational speech.

\subsection{Knowledge Acquisition}
To evaluate the overall effectiveness of the educational intervention, we compared the aggregated knowledge accuracy between the Pre-survey (N = 61) and Post-survey (N = 32) phases. Due to the anonymous nature of the survey collection which prevented paired data matching, a group-level analysis was conducted using an independent samples t-test.
\begin{table}[h]
\centering
\caption{Comparison of Knowledge Accuracy Between Pre-Survey and Post-Survey}
\label{tab:pre_post_accuracy}
\footnotesize
\setlength{\tabcolsep}{2pt}
\begin{tabular}{lccccccc}
\toprule
\textbf{Group} 
& \textbf{N} 
& \textbf{Mean Accuracy (\%)} 
& \textbf{SD (\%)} 
& \textbf{$t$} 
& \textbf{df} 
& \textbf{$p$} 
& \textbf{Cohen's $d$} \\
\midrule
Pre-Survey 
& 61 & 69.18 & 32.16 & -1.35 & 70.82 & .181 & 0.29 \\
Post-Survey 
& 32 & 78.13 & 29.34 &  &  &  &  \\
\bottomrule
\end{tabular}
\end{table}
\\
Table~\ref{tab:pre_post_accuracy} presents the descriptive statistics and t-test results for the knowledge assessment. The Post-survey group demonstrated a higher mean accuracy (M = 78.13\%, SD = 29.34\%) compared to the Pre-survey baseline (M = 69.18\%, SD = 32.16\%). Although this difference did not reach statistical significance (t(70.82) = -1.35, p = .181), the analysis revealed a small-to-medium effect size (Cohen's d = 0.29), suggesting a positive trend in knowledge retention following the avatar-based learning session.

\section{Discussion}
\subsection{Interpretation of Results}
This study examined how empathic versus neutral AI avatar personalities influence learner engagement and perceived learning in an AI-supported educational environment. While no statistically significant differences were observed in learning scores, sentiment, confusion rates, or response times between the two conditions, clear differences emerged in interaction behavior.
\\
Participants interacting with the empathic avatar produced longer responses and engaged in more conversational turns compared to those interacting with the neutral avatar. These findings suggest that avatar empathy may influence how learners engage with the system, even when measurable short-term learning outcomes remain comparable. In other words, while empathy did not directly translate into higher test performance within this short intervention, it appeared to shape the interaction dynamics of the learning experience.
\\
One possible interpretation is that empathic language may be perceived as more supportive or socially responsive, which could encourage learners to elaborate more extensively in their responses. However, psychological comfort or perceived social presence were not directly measured in this study; therefore, this interpretation remains speculative and should be examined in future research.
\\
Interestingly, although empathic interactions were characterized by greater response length and conversational density, time spent in the system was not strongly associated with measurable learning gains. This suggests that extended interaction alone does not necessarily produce improved short-term knowledge retention. Instead, engagement and learning performance may represent related but distinct dimensions of AI-supported education.
\\
Overall, the findings indicate that empathic AI tutoring may function more as an engagement catalyst rather than a direct driver of short-term measurable learning performance. This distinction is important when evaluating the effectiveness of emotionally expressive AI systems.


\subsection{Implications for Human–AI Interaction Design}
The results provide several implications for the design of Human–AI Interaction in educational contexts.
\\
First, the findings suggest that emotional tone in AI tutors can meaningfully shape user interaction behavior. Even in the absence of statistically significant differences in learning outcomes, empathic design appears to influence how users communicate with the system. Designers of educational AI systems should therefore consider engagement-related metrics—such as conversational depth, response length, and interaction patterns—alongside traditional performance indicators.
\\
Second, the results indicate that affective design alone may not be sufficient to enhance short-term learning outcomes. In this study, empathy did not significantly improve knowledge test scores compared to a neutral instructional style. This suggests that emotional expressiveness should be integrated with structured pedagogical strategies, scaffolding techniques, or adaptive feedback mechanisms to produce measurable learning gains.
\\
Third, the study highlights the importance of distinguishing between engagement and effectiveness. An AI tutor may successfully increase user interaction and conversational participation without necessarily improving immediate test performance. For researchers and practitioners, this distinction emphasizes the need for multi-dimensional evaluation frameworks when assessing AI-supported learning systems.
\\
Finally, the findings contribute to ongoing discussions in HAI research about the role of social presence in AI systems. While emotionally expressive avatars may enhance user engagement, their impact on cognitive learning outcomes may depend on contextual variables such as intervention duration, task complexity, and learner characteristics.

\subsection{Limitations}
Several limitations should be considered when interpreting the findings of this study.
First, the sample size was relatively small (N = 30), which limits statistical power and may reduce the likelihood of detecting subtle effects. Larger samples would allow for more robust statistical testing and subgroup analyses.
\\
Second, the intervention was short-term and limited to introductory psychology topics. The absence of significant learning differences between conditions may reflect the brief exposure duration rather than the ineffectiveness of empathic design. Longitudinal studies are needed to examine whether empathy influences knowledge retention over extended periods.
\\
Third, the study relied primarily on self-reported measures and short-term knowledge tests. While these instruments are commonly used in HAI and educational research, they may not fully capture deeper cognitive processing, long-term retention, or transfer of learning.
\\
Fourth, participants were mostly university students with prior experience using AI tools. This familiarity may have influenced expectations and interaction styles, potentially limiting the generalizability of the findings to other populations.
\\
Finally, although the system is controlled for teaching content across conditions, subtle variations in interaction trajectories may naturally emerge in dialog-based LLM systems. While prompt engineering ensures structural consistency, real-time conversational dynamics may introduce variability that is difficult to standardize fully.
\\
Taken together, these limitations suggest that the findings should be interpreted as exploratory evidence regarding engagement effects rather than definitive conclusions about the role of empathy in AI-supported learning.



\section{Conclusion}
\subsection{Summary of Findings}

\subsection{Future Work}
Future research should address several existing research gaps in the application of LLM-based pedagogical agents in education. While current studies demonstrate positive short-term effects on learning outcomes, there is still limited evidence regarding the long-term impact of these systems on knowledge retention, critical thinking development, and learner independence. 
\\
In addition, although social cues in affective pedagogical agents have been shown to improve engagement and motivation, there is a lack of research identifying the optimal design and level of human-likeness required for different learner groups and educational contexts. 
\\
Additionally, a notable gap in research is the scarcity of practical implementations in real classroom settings, as many experiments are still performed in controlled environments. Consequently, future research should prioritize studies conducted in actual educational contexts, create standardized assessment models for Human–AI educational technologies, and address ethical issues such as protecting data privacy, ensuring transparency, and promoting responsible AI usage to support a sustainable and reliable integration of technology in education.



%%
%% The next two lines define the bibliography style to be used, and
%% the bibliography file.
\bibliographystyle{ACM-Reference-Format}
\bibliography{sample-base}


%%
%% If your work has an appendix, this is the place to put it.
\appendix

\section*{Appendix: Project Repository}

The full versions of the pre-survey and post-survey questionnaires are available online:

\href{https://docs.google.com/document/d/1IAUKnw7fmOBHQN_dgI0UUycK-XqBsrUmwkp1BBnLBTI/edit?usp=sharing}{Pre-survey}

\href{https://docs.google.com/document/d/1cYRWFesFMhTQ-Egog53PdmHoVywXFJS5jFVYG6f0BXY/edit?usp=sharing}{Post-survey}

The full implementation of the system, including version history and contribution records, is available at:

\href{https://experiment-ghbdjy5jwhtgvmb7efhmyc.streamlit.app}{Psychology Learning Experiment}

\href{https://github.com/yiyangDelix/haii}{GitHub from Yiyang Xie}

\href{https://github.com/yusongyangtum-yys/Experiment}{GitHub from Yusong Yang}
\section*{Contribution Overview}

The following list summarizes the main contributions of each group member throughout the project.

\subsection*{Araks Karapetyan}
\begin{itemize}
    \item Shared workspace \& group coordination: Organized internal group workflow, established shared online workspace (Word, PPT), and coordinated group communication.
    \item Contributed to the preparation of presentation slides.
    \item Designed learning scenario 1 and 2, including learning content, prompts, and quizzes.
    \item Contributed to the midterm report preparation, participated to all the team meetings. 
    \item Designed the post-study questionnaire.
    \item Tested the avatar-based learning system and surveys.
    \item Contributed in finding the research question and hypotheses for the final report.
    \item Participated in recruiting participants for the user study. (12 people)
    \item Designed the research question and hypotheses for the final report, while also writing the whole “Methodology” section and the “Abstract”.
    \item Created the “Pre-survey” and “Post-survey” questionnaire files for the final report’s Appendix.
\end{itemize}


\subsection*{Brishila Firza}
\begin{itemize}
    \item Tested the avatar-based learning system and surveys and provided feedback on the interaction flow.
    \item Contributed to the ideation of survey questions by providing feedback.
    \item Contributed to participant recruitment and data collection. (9 people)
    \item Participated in all group meetings and contributed to group organization and shared workflow, including collaborative documents and communication.
    \item Solely presented the pitch presentation.
    \item Contributed to the midterm report preparation.
    \item Designed detailed learning scenarios (content, prompt \& quizzes) for the avatar modes and conducted a literature review. 
    \item Video creation and upload: Solely drafted, created and edited the project video, and uploaded the final version to Moodle.
    \item Participated in experiments for bonus engagement points.
    \item Contributed in finding the research question and hypotheses for the final report.
    \item Prepared 6 sections of the Final Report : 1.3 , 2.1 , 2.2 , 2.3, 7.1, 7.2 and added their related references to the appendix.
\end{itemize}


\subsection*{Nergis Bilge}
\begin{itemize}
    \item Participated in all team meetings and contributed to shared workflow and group coordination.
    \item Contributed to the preparation of presentation slides and the midterm report.
    \item Narrowed learning topics from six to three core psychology concepts for the study questionnaires.
    \item Tested the avatar-based learning system and provided structured feedback on interaction flow and usability.
    \item Designed and refined the pre-study questionnaire, including learning assessment and Likert-scale items.
    \item Aligned the post-study questionnaire with the pre-study learning measures to ensure consistency.
    \item Contributed to participant recruitment and data collection (9 participants).
    \item Assisted with data preparation and cleaning, identifying valid responses for analysis.
    \item Prepared parts of the final report, including the Introduction (1.1–1.2) and Discussion (6.1–6.3).
\end{itemize}


\subsection*{Yiyang Xie}
\begin{itemize}
    \item Deployed the LLM remotely, configured API access, and integrated the chatbot with the 3D avatar in a web-based interface.
    \item Implemented the initial (1st. version) learning system architecture.
    \item Designed the basic components for the system, including two types of chatbot, system prompts, and avatar design.
    \item Performed system debugging and performance optimization.
    \item Designed learning scenarios 4, 5 and 6, including learning content, prompts, and quizzes.
    \item Contributed to the preparation of presentation slides.
    \item Participated in recruiting participants for the user study. (7 people)
    \item Designed overall report structure, discussed adjustments with supervisor.
    \item Drafted full text version for midterm and final reports, authored Section 3 of the final report, and implemented both reports using \LaTeX.
\end{itemize}


\subsection*{Yusong Yang}
\begin{itemize}
    \item Further developed the learning system and extended its functionality.
    \item Implemented speech-based interaction, enabling voice input and output.
    \item Designed and implemented automatic user ID generation, data linking and synchronization of user study data to a shared Excel file.
    \item Integrated pre-test and post-test into the learning system.
    \item Contributed to the preparation of presentation slides.
    \item Contributed to the midterm report preparation.
    \item Performed system debugging and performance optimization.
    \item Designed the automated data logging structure, ensuring collection of behavioral metrics such as duration, word count, response time, sentiment score, confusion rate, and turn count.
    \item Participated in recruiting participants for the user study.
    \item Implemented the data analysis and authored Section 5 of the final report.
\end{itemize}



\end{document}
\endinput
%%
%% End of file `sample-sigplan.tex'.
